%! TEX root = ./{filename}
\documentclass{article}
\usepackage{amsmath}
\usepackage{amsfonts}
\usepackage{amsthm}

\title{A Mathematical Framework for Music}
\author{}
\date{}

\begin{document}

\maketitle

\section{Foundations}
It starts with a single note, which we shall call $A_1$, vibrating at a frequency $X$. Shortly after, a note $A_2$ follows, vibrating at a frequency $2X$. The difference between $A_1$ and $A_2$ is an \textbf{octave}.

We find notes with simple frequency ratios (small integer numerators and denominators) pleasing to the ear.


\begin{equation}
    q = 2^{1/n}
\end{equation}

Where $n$ is the number of notes per octave. Any note on the instrument can be located within a geometric series, one with ratio $q$ and one with ratio $1/q$. Each octave on the instrument corresponds to one such geometric series, and the set of all octaves forms the complete set of such series.


It will follow that q is a semi-tone, and that the first number in such a series, $A_1$ is the tonal center. our "base note".

\section{Notes and Scales}
Each note's frequency is given by:

\begin{equation}
    f_k = A \cdot q^k
\end{equation}

where $k$ is the note index. 

Scales are subsets of this sequence. For example, in a 12-tone equal temperament system, the major scale consists of the indices:

\begin{equation}
    \{0, 2, 4, 5, 7, 9, 11\}
\end{equation}

relative to the starting note.

\section{Harmonic Intervals}
Two notes $f_k$ and $f_m$ form a ratio:

\begin{equation}
    r = \frac{f_m}{f_k} = q^{m-k}
\end{equation}

which can be approximated by simple fractions:

\begin{itemize}
    \item Octave: $2/1$
    \item Perfect Fifth: $3/2$
    \item Perfect Fourth: $4/3$
    \item Major Third: $5/4$
    \item Minor Third: $6/5$
\end{itemize}

So, for example:

We know that \( 1.059 \approx 2^{1/12} \).

\[
3/2 = 1.5 = (1.059)^{m-k}
\]

Taking the logarithm of both sides:

\[
\log(1.5) = (m-k) \cdot \log(1.059)
\]

Solving for \( m-k \):

\[
m-k \approx \frac{\log(1.5)}{\log(1.059)} \approx 7
\]
 
Which is perfectly consistant with reality, given that it's exactly 7 half steps away on a piano!


\section{Chords as Frequency Sets}
A chord is a set of notes whose frequencies maintain harmonic relationships. For example:

\begin{itemize}
    \item \textbf{Major Chord} (C-E-G): Frequencies in $4:5:6$ ratio.
    \item \textbf{Minor Chord} (C-E$\flat$-G): Frequencies in $10:12:15$.
    \item \textbf{Diminished Chord} (C-E$\flat$-G$\flat$): Frequencies in $160:192:231$.
\end{itemize}

\section{Modulation as a Transform}
Modulation (changing keys) can be described as shifting all notes by $m$ semitones:

\begin{equation}
    f_k' = A \cdot q^{k+m}
\end{equation}

which is equivalent to multiplying all frequencies by $q^m$.

\section{Definitions of Musical Terms}
\textbf{Key}: A key defines the tonal center and set of notes forming a scale in a piece of music.

\textbf{Consonance}: The quality of notes sounding stable or pleasant together, often corresponding to simple frequency ratios.

\textbf{Dissonance}: The quality of notes sounding unstable or tense, often corresponding to more complex frequency ratios.

\textbf{Perfect Fifth}: The interval between two notes with a frequency ratio of $3/2$. It is called ``perfect'' due to its strong harmonic stability and is one of the fundamental building blocks of Western harmony.

\end{document}
