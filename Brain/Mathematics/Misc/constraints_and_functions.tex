%! TEX root = ./{filename}
\documentclass{article}
\usepackage{amsmath}
\usepackage{amsfonts}
\usepackage{amsthm}

\begin {document}

\section{Why?}

Because it stroke me as interesting in math class! hehe.

Also, I might have misused the term "well-defined", not sure.

\section*{Function Composition and Constraints via Systems of Equations}

Let \( n \) functions \( f_1, f_2, \dots, f_n \) be defined as:
\[
f_i: X_i \to X_{i+1}, \quad \text{for } i = 1, 2, \dots, n,
\]
where \( X_i \) denotes the domain of \( f_i \), and \( X_{i+1} \) denotes its codomain.

\subsection*{Composite Function:}
The composite function \( h \) is defined as:
\[
h(x) = f_n(f_{n-1}(\cdots f_1(x) \cdots)),
\]
where \( x \in X_1 \). The evaluation of \( h(x) \) imposes the following constraints:
\begin{enumerate}
    \item \( x \in X_1 \), such that \( f_1(x) \) is well-defined.
    \item \( f_1(x) \in X_2 \), such that \( f_2(f_1(x)) \) is well-defined.
    \item More generally, for \( i = 1, 2, \dots, n-1 \):
    \[
    f_i(f_{i-1}(\cdots f_1(x) \cdots)) \in X_{i+1}.
    \]
    \item Finally, \( f_n(f_{n-1}(\cdots f_1(x) \cdots)) \in X_{n+1} \).
\end{enumerate}

\subsection*{Equivalent System of Equations:}
The constraints imposed by \( h(x) \) are equivalent to solving the system of equations:
\[
\begin{aligned}
y_1 &= f_1(x), \\
y_2 &= f_2(y_1), \\
y_3 &= f_3(y_2), \\
&\vdots \\
y_n &= f_n(y_{n-1}),
\end{aligned}
\]
where \( y_i \in X_{i+1} \) for \( i = 1, 2, \dots, n \). The solution set to this system is the same as the set of inputs \( x \in X_1 \) for which \( h(x) \) is defined and satisfies all constraints. This is because we can substitute each variable $Y_i_$ into the next equation.

\subsection*{Generalization:}
For any finite sequence of \( n \) functions \( f_1, f_2, \dots, f_n \), the constraints imposed by their composition can always be expressed as a system of \( n \) equations:
\[
y_i = f_i(y_{i-1}), \quad \text{where } y_0 = x \text{ and } i = 1, 2, \dots, n.
\]
Each equation in the system introduces a new constraint that must be satisfied for \( x \) to be a valid input to the composed function.

Therefore, The process of composing functions inherently ties the output of one function to the domain of the next, effectively constraining the system to values that align with both functions' rules!

\section*{Properties of Function Composition and Systems of Equations}

\subsection*{One-to-One Mapping}

So as we said: 
For every valid \( x \) in the domain of the composition, there is exactly one corresponding solution to the system of equations:
\[
\begin{aligned}
y_1 &= f_1(x), \\
y_2 &= f_2(y_1), \\
y_3 &= f_3(y_2), \\
&\vdots \\
y_n &= f_n(y_{n-1}).
\end{aligned}
\]
Here, \( y_i \in X_{i+1} \) for \( i = 1, 2, \dots, n \), and \( h(x) = f_n(f_{n-1}(\dots f_1(x) \dots)) \) is the final output.

Similarly, every solution of the system maps back to a valid evaluation of the composite function \( h(x) \), ensuring a one-to-one correspondence between the two representations.

\subsection*{Reversibility}
The process of converting a function composition into a system of equations is invertible, meaning the two representations are structurally equivalent. Specifically:
\begin{itemize}
    \item Given the composite function \( h(x) \), the equivalent system of equations can be constructed by breaking down each step of the composition.
    \item furthermore, given the system of equations, the composite function \( h(x) \) can be recovered by substituting sequentially through the equations:
    \[
    h(x) = f_n(f_{n-1}(\dots f_1(x) \dots)).
    \]
\end{itemize}

Thus, the equivalence ensures that the two frameworks describe the same relationships between inputs and outputs.

\subsection*{1. Associativity of Function Composition}
Let \( f: X \to Y \), \( g: Y \to Z \), and \( h: Z \to W \). Then:
\[
(h \circ g) \circ f = h \circ (g \circ f).
\]

\paragraph{Proof:} Using the system of equations:
\begin{itemize}
    \item For \( (h \circ g) \circ f \):
    \[
    \begin{aligned}
    y_1 &= f(x), \\
    y_2 &= g(y_1), \\
    y_3 &= h(y_2).
    \end{aligned}
    \]
    \item For \( h \circ (g \circ f) \):
    \[
    \begin{aligned}
    y_1 &= f(x), \\
    y_2 &= g(y_1), \\
    y_3 &= h(y_2).
    \end{aligned}
    \]
\end{itemize}
In both cases, \( y_3 \) is computed identically, proving that \( (h \circ g) \circ f = h \circ (g \circ f) \).

\subsection*{2. Identity Function}
Let \( f: X \to Y \), and let \( \text{id}_X(x) = x \) and \( \text{id}_Y(y) = y \) be the identity functions on \( X \) and \( Y \), respectively. Then:
\[
f \circ \text{id}_X = f \quad \text{and} \quad \text{id}_Y \circ f = f.
\]

\paragraph{Proof:}
\begin{itemize}
    \item For \( f \circ \text{id}_X \):
    \[
    \begin{aligned}
    y_1 &= \text{id}_X(x) = x, \\
    y_2 &= f(y_1) = f(x).
    \end{aligned}
    \]
    Thus, \( f \circ \text{id}_X = f \).
    \item For \( \text{id}_Y \circ f \):
    \[
    \begin{aligned}
    y_1 &= f(x), \\
    y_2 &= \text{id}_Y(y_1) = y_1 = f(x).
    \end{aligned}
    \]
    Thus, \( \text{id}_Y \circ f = f \).
\end{itemize}

\subsection*{3. Domain and Range Constraints}
The domain of \( f \circ g \) is constrained by both \( g \) and \( f \):
\[
\text{Domain}(f \circ g) = \{ x \in \text{Domain}(g) \mid g(x) \in \text{Domain}(f) \}.
\]

\paragraph{Proof:}
\begin{itemize}
    \item \( y_1 = g(x) \) is well-defined if \( x \in \text{Domain}(g) \).
    \item \( y_2 = f(y_1) \) is well-defined if \( y_1 \in \text{Domain}(f) \).
\end{itemize}
Thus, \( f \circ g \) is defined only when \( g(x) \in \text{Domain}(f) \), which constrains \( x \) to the given set.

\section{Is this useful?}

This might be completely elementary. I don't know enough math to say. But to me, it looked interesting and fun to take a look at.



\end{document}

