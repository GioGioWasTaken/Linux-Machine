%! TEX root = ./statement_holds.tex
\documentclass{article}
\usepackage{amsmath}
\usepackage{amsfonts}
\usepackage{amsthm}

\begin {document}

\section{Preface}

So I had this question in my mind today, of whether or not it can be proven that if a theorem about some operation holds true for 2 operands, it also holds true for infinitely many operands.

Upon some thinking the answer I reached was hell no!

So, what else is required, then?

\section{Proof}

Let \( H \) be a set with an associative operation \( * \) such that for all \( a, b, c \in H \), we have
\[
(a * b) * c = a * (b * c).
\]
Suppose also that for all \( a, b \in H \), the operation satisfies \( a * b = d \) for some fixed element \( d \in H \).

We aim to show by induction that for any \( N \)-operand expression of the form
\[
a * b * c * \dots * z,
\]
where \( z \) is the \( N \)-th operand, the result of the operation is \( d \).

\subsection{Base Case}
For the base case of two operands, we are given that for any \( a, b \in H \), 
\[
a * b = d.
\]
Thus, the base case holds.

\subsection{Inductive Hypothesis}
Assume that for any \( k \) elements \( a_1, a_2, \ldots, a_k \in H \), we have
\[
a_1 * a_2 * \dots * a_k = d.
\]
This is our inductive hypothesis.

\subsection{Inductive Step}
We must show that if the hypothesis holds for \( k \) operands, then it also holds for \( k+1 \) operands. That is, we want to prove
\[
a_1 * a_2 * \dots * a_k * a_{k+1} = d.
\]

Using associativity, we can group the expression as follows:
\[
a_1 * a_2 * \dots * a_k * a_{k+1} = (a_1 * a_2 * \dots * a_k) * a_{k+1}.
\]
By the inductive hypothesis, we know that \( a_1 * a_2 * \dots * a_k = d \). Substituting this into the expression, we get:
\[
(a_1 * a_2 * \dots * a_k) * a_{k+1} = d * a_{k+1}.
\]
Finally, by the given assumption that \( a * b = d \) for all \( a, b \in H \), we know that \( d * a_{k+1} = d \).

Thus, we have shown:
\[
a_1 * a_2 * \dots * a_k * a_{k+1} = d.
\]

\subsection{Conclusion}
By induction, we have proven that for any \( N \)-operand expression \( a * b * c * \dots * z \) with \( a, b, c, \dots, z \in H \), the result of the operation is \( d \). Therefore,
\[
a * b * c * \dots * z = d
\]
for any finite number of operands in \( H \), which completes the proof.

\subsection{Multiplication of Complex Numbers in Polar Form}
Let us also consider the multiplication of two complex numbers in polar form. 

If we have two complex numbers given by:
\[
z_1 = r_1 \text{cis}(\theta_1) \quad \text{and} \quad z_2 = r_2 \text{cis}(\theta_2),
\]
where \( r_1 \) and \( r_2 \) are the magnitudes (moduli) and \( \theta_1 \) and \( \theta_2 \) are the angles (arguments), the multiplication is given by:
\[
z_1 * z_2 = (r_1 \text{cis}(\theta_1)) * (r_2 \text{cis}(\theta_2)) = r_1 r_2 \text{cis}(\theta_1 + \theta_2).
\]

Now this by itself is quite useful, but given multiplication is associative in $\mathbb{C}$ as it is a field, we can now multiply however many complex numbers we like!


\end {document}

