%! TEX root = ./Z_construction.tex
\documentclass{article}
\usepackage{amsmath}
\usepackage{amsfonts}
\usepackage{amsthm}

\begin {document}


\section{Preface}
The set of integers $\mathbb{Z}$ can be constructed from the set of natural numbers $ \mathbb{N}$ by introducing the concept of ordered pairs to represent negative numbers and zero.

We let $R \subseteq N\times N $ be an equivalence relation on it. Defined as:
\\
$
(a,b)R(c,d) \iff a-b = c-d
$
\\
\\
Which neccesitates:
\\
$
-1 := \{ (2,3), (3,4), \dots, (n, n+1) \}
$
\\

Similarily:
\\
1 := \{ (3,2), (4,3), \dots, (n+1, n) \}

\section{Operations}

\subsection{Addition}

Therefore, It will be defined using the equivalence relationship previously established, that:

$
a,b \in Z,$ and $p,q,m,c\in \mathbb{N}$ s.t. $a=(p,q), b=(m,c) \subseteq R$  a+b= $(p+m, q+c)$
$

Since: 

$
(a,b) +(c,d) = (a-b) + (c-d) = a+b -(c+d)
$

\subsection{Multiplication}

Similarily:

$(p-q) \cdot (m-c) = (pm-pc-qm+qc) = (pm+qc) -(pc+qm) $
\\
Therefore:

$ a\cdot b = (pm+qc, pc+qm)$

$ \qed $
\end{document}
