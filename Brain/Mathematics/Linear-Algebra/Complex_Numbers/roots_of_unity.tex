%! TEX root = ./{filename}
\documentclass{article}
\usepackage{amsmath}
\usepackage{amsfonts}
\usepackage{amsthm}

\begin {document}

\section{Contents}

I will prove the sum of the Nth roots of unity is 0 for any n.

\section{Proof}

So we will say that $\zeta_1$ is the first Nth root of unity if and only if:
$\zeta_1^n =1$.\bigskip

It is known that for any complex number z, there are n Nth roots. Therefore, this also holds for 1.

So there are n Nth roots of unity. We will try to show that for the roots of unity their sum is always 0.
We can find the Nth roots of a complex number Z by this formula:

$
\sqrt[n]{z} = \sqrt[n]{r} \cdot \text{cis} \left( \frac{\theta + 2k\pi}{n} \right)
$
The polar represenation of 1 is $ 1cis(0) $. So this simplifies to:

$ \sqrt[n]{1} = 1\cdot \text{cis} \left( \frac{2k\pi}{n} \right)$

It is proven that $ \frac{cis(a)}{cis(b)} = cis(a-b)$. Therefore:

$
\frac{cis(\frac{2k\pi}{n})}{cis(\frac{2(k+1)\pi}{n})}=
cis(\frac{2k\pi}{n} - (\frac{(2k+2)\pi} {n})) =
cis(\frac{2\pi}{n})
$

So in general the w Nth root of unity is: 

$ a_w =  cis(0)\cdot cis(\frac{2\pi}{n})^{w-1} $



\end {document}

